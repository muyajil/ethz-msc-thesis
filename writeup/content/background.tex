\chapter{Background}

\section{Recommendation Systems}
\subsection{Problem Statement}
\begin{itemize}
    \item The general form of a recommendation system is that we assign scores to items.
    \item A personalized recommendation system uses the context of the user to produce scores dependant of the active user.
    \item Write out a function that describes a general recommendation system
\end{itemize}
\subsection{Variants}
\begin{itemize}
    \item User Based
    \item Item Based
    \item Session Based
\end{itemize}
\subsection{Well-Known Systems}
\begin{itemize}
    \item Describe a classical user based approach
    \item Describe a classical item based approach
\end{itemize}

\section{Concepts}
\subsection{Recurrent Neural Networks}
\begin{itemize}
    \item Explain the concept of neural Networks
    \item What are the problems (vanishing gradients etc)
    \item Explain what a GRU is and why does it handle vanishing gradient better than simple RNN cells?
    \item How does a layered RNN work?
\end{itemize}
\subsection{Generative Adversarial Network Framework}
\begin{itemize}
    \item Explain the way GANs work
    \item Loss function is learnable
    \item The discriminator is part of the loss function
    \item Especially good if we want generate stuff
\end{itemize}
\subsection{Teacher Forcing}
\begin{itemize}
    \item RNN have two modes: Teacher forced and Free running
    \item When training we always use the ground truth as input
    \item This generates two different distributions of hidden states, which in turn generates two different distributions for the output
    \item This is a problem since this makes RNNs in production less predictable
\end{itemize}

\section{Previous Work}
\subsection{Professor Forcing}
\begin{itemize}
    \item Here we want to mitigate the teacher forcing problem, by applying the gan framework
    \item Add graphic showing how it should work
    \item Describe the model as we did in the slides, show the loss function
\end{itemize}
\cite{profforce}
\subsection{Meta-Prod2Vec}
\begin{itemize}
    \item This is a model that allows us to capture the semantic similarity between products
    \item Explain how it works by relating to word2vec (refer to word2vec paper)
\end{itemize}
\cite{prod2vec}
\subsection{Hierarchical RNNs for personalized Recommendations}
\begin{itemize}
    \item This is the starting point for the thesis
    \item Which components does the system have?
    \item Show the graphic from the paper
\end{itemize}
\cite{hierarchical}

\section{KPIs}
\subsection{Click-Through-Rate}
\subsection{Conversion Rate}
\begin{itemize}
\item Describe different KPIs
\item What do they meaure, how to optimize for it
\end{itemize}