\chapter{Background}

\section{Recommendation Systems}
\subsection{Problem Statement}
Generally speaking recommendation systems are concerned with recommending items to be of use to users.
The recommendations should help the users decide which items to buy, music to listen to, what news articles to read etc.
Virtually any decision-process can be made easier for users by providing recommendations.
Typically recommendation systems are used by online services, famously for example by Spotify for music~\cite{rec_spotify} and YouTube for videos~\cite{rec_yt}.
By using these online service the users generate data that the service can use to improve the recommendations, for example rating videos gives the operator of the service a datapoint on how much a video is liked by a specific user.
However also retailers are known to use recommendation systems, based on data generated by customer loyalty programs such as Migros Cumulus~\cite{rec_migros} retailers tailor coupons or other offers to their customers.
In principle data of users interacting with items (viewing, buying, rating etc.) serves as the basis for a recommendation system. 
Depending on the use-case this data is utilized to assign scores to items depending on the user. 
The semantic meaning of a score depends on the objective of a recommendation system.
For example a system which recommends products to be bought might assign a "probability of purchase", whereas a system which recommends videos might predict the probability of a user watching a video to the end. 
\\
Bringing this all together we can define a general recommendation system with the following function:
\[
    s = f(i, u, h_u)
\]
Where $s$ is the assigned score to item $i$ for the user $u$ and the users history $u_h$. 
The users history represents all the previous interactions with items.
Essentially when we design a recommendation engine we want to learn the function $f$.
To assess the quality of the learned function we need to know the "true" score for the specific item and user, this can be done in different ways.
Usually during training we will hold off later interactions with items, and test the learned functions on those.
However as soon as the users sees recommendations, we essentially change the reality, i.e. we don't know what the user saw as recommendations in the user history, therefore it makes sense to also assess the performance of a recommendation engine in production.
\begin{itemize}
    \item The general form of a recommendation system is that we assign scores to items.
    \item A personalized recommendation system uses the context of the user to produce scores dependant of the active user.
    \item Write out a function that describes a general recommendation system
\end{itemize}
\subsection{Variants}
\begin{itemize}
    \item User Based
    \item Item Based
    \item Session Based
\end{itemize}
\subsection{Well-Known Systems}
\begin{itemize}
    \item Describe a classical user based approach
    \item Describe a classical item based approach
\end{itemize}

\section{Concepts}
\subsection{Recurrent Neural Networks}
\begin{itemize}
    \item Explain the concept of neural Networks
    \item What are the problems (vanishing gradients etc)
    \item Explain what a GRU is and why does it handle vanishing gradient better than simple RNN cells?
    \item How does a layered RNN work?
\end{itemize}
\subsection{Generative Adversarial Network Framework}
\begin{itemize}
    \item Explain the way GANs work
    \item Loss function is learnable
    \item The discriminator is part of the loss function
    \item Especially good if we want generate stuff
\end{itemize}
\subsection{Teacher Forcing}
\begin{itemize}
    \item RNN have two modes: Teacher forced and Free running
    \item When training we always use the ground truth as input
    \item This generates two different distributions of hidden states, which in turn generates two different distributions for the output
    \item This is a problem since this makes RNNs in production less predictable
\end{itemize}

\section{Previous Work}
\subsection{Professor Forcing}
\begin{itemize}
    \item Here we want to mitigate the teacher forcing problem, by applying the gan framework
    \item Add graphic showing how it should work
    \item Describe the model as we did in the slides, show the loss function
\end{itemize}
\cite{profforce}
\subsection{Meta-Prod2Vec}
\begin{itemize}
    \item This is a model that allows us to capture the semantic similarity between products
    \item Explain how it works by relating to word2vec (refer to word2vec paper)
\end{itemize}
\cite{prod2vec}
\subsection{Hierarchical RNNs for personalized Recommendations}
\begin{itemize}
    \item This is the starting point for the thesis
    \item Which components does the system have?
    \item Show the graphic from the paper
\end{itemize}
\cite{hierarchical}

\section{KPIs}
\subsection{Click-Through-Rate}
\subsection{Conversion Rate}
\begin{itemize}
\item Describe different KPIs
\item What do they meaure, how to optimize for it
\end{itemize}