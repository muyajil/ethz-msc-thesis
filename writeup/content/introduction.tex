\chapter{Introduction}

\begin{itemize}
\item Recommender System are a field in machine learning that get more and more attention
\item In principle the goal of a recommender system is to recommend items (in the case of an e-commerce system products) to users
\item The idea is that the system helps the user navigate the mass of products available and to show the user what is relevant
\item A recomender system can be tuned to optimize for different metrics such as click through rate, conversion rate etc.
\item In the past there have been mainly to approaches to recommender systems user based and item based
\item In the last years there have been more and more proposals for session based approaches (refer to paper that does the survey)
\item Specifically GRU4Rec has gained attention as a RNN based approach to solving this problem
\item In a session based setting we try to model the sequence of events a user makes when he browses the product catalog instead of items or users themselves.
\item This has the following advantages: Usually we have a lot of users that only visit the site once or twice a year, these users are very difficult to model. The same goes for products, there are a lot of products that have limited attention from the userbase, therefore it is difficult to get a useful representation for products like this. 
\item In a session based setting we model the sequence of events, which is more useful since there is a lot more data, and sessions are comparable even if we do not know which user is online.
\item Session based approaches work in both settings where we dont know the user and where we know the user
\item Item based are for unkown users and user based for known users
\item The goal of the thesis is twofold: 
\begin{itemize}
    \item First we want to explore the capabilities of such a system in a production setting with real world data. Are we really better than simple approaches like "often bought together"? 
    \item Also we want to compare it to a more sophisticated system that is a blackbox and consumed as a service provided by Google.
    \item Third, since this is an RNN approach we want to find out if we can improve the system by applying the GAN framework in form of professor forcing
    \item Fourth, the system inherently produces embeddings for products and users, we want to find out if we can improve the overall performance if we use a pretrained product embedding which captures the semantic similarity of products
\end{itemize} 
\end{itemize}